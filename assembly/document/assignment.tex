\documentclass{article}
\usepackage[none]{hyphenat}
\usepackage{enumitem}
\usepackage{graphics}
\usepackage{graphicx}
\usepackage{ragged2e}
\usepackage{multirow}
\usepackage{blindtext}
\usepackage{amsmath}
\usepackage{subcaption}
\usepackage{circuitikz}
\usepackage{listings}
\usepackage{./karnaugh-map}
\usetikzlibrary{arrows,shapes,automata,petri,positioning,calc}
\usetikzlibrary{shapes.geometric}
\lstset{
	language=C++,
	basicstyle=\ttfamily\footnotesize,
	breaklines=true,
	frame=lines
	}
\title{Implementation of finite state machine}
\date{March 2023}
\author{M Sai Sarath Chandra\\chandu.4567890@gmail.com\\FWC22117\\IIT Hyderabad-Future Wireless Communication Assignment}

\begin{document}
\maketitle
	\tableofcontents

\pagebreak
\section{Problem}
	{GATE EC-2020}\\
	Q.39. The state diagram of a sequence detector is shown below.State S0 is the initial state of the seuence detector.If the output is 1,then 
\\
\begin{figure}[h]
	\centering
\begin{figure}[h]
	\begin{center}
\begin{tikzpicture}
\ctikzset
{
logic ports=ieee,
logic ports/scale=0.8
}
\draw(2,1) -- (4,1);
\draw(2,0) -- (4,0);
\draw(2,-1) -- (4,-1);
\draw(2,-2) -- (4,-2);
\draw(4,-3) -- (4,2);
\draw(4,2) -- (7,2);
\draw(7,2) -- (7,-3);
\draw(7,-3) -- (4,-3);
\draw(5,-3) -- (5,-5);
\draw(6,-3) -- (6,-5);
\draw(7,-0.5) -- (9,-0.5);
\draw(1.5,1) node{$W$};
\draw(1.5,0) node{$X$};
\draw(1.5,-1) node{$Y$};
\draw(1.5,-2) node{$Z$};
\draw(5,-5.5) node{$R$};
\draw(6,-5.5) node{$S$};
\draw(9.5,-0.5) node{$A$};
\end{tikzpicture}
	\end{center}
\caption{mux}
\label{fig:1}
\end{figure}

	\caption{FSM}
	\label{fig:1}
\end{figure}
\\
\begin{enumerate}
	\item the sequence 01010 is detected.
	\item the sequence 01011 is detected.
	\item the sequence 01110 is detected.
	\item the sequence 01001 is detected.
\end{enumerate}

\section{Introduction}
		The aim is to implement the above finite state machine using 7447 and 7474 IC's.Finite state machine is a device which stores a state at a given time,the state will then change based on inputs,provding the resulting output for the implemented changes.IC 7474 is a dual positive-edge-triggered D-type flip flop,which means it has two separate flip-flop that are triggered by the rising edge of a clock signal.

\section{Components}
	\begin{table}[h]
		\begin{center}
			\begin{tabular}{|c|c|c|}
\hline
Symbol & Value & Description\\
\hline
$x$ & 16cm & $\vec{AB}$ \\
\hline
$a$ & 10cm & $\vec{CF}$ \\
\hline
$b$ & 8cm & $\vec{AE}$ \\
\hline
$\angle{CFD}$ & $90\degree$ & $CF \perp AD$ \\
\hline
$\angle{AED}$ & $90\degree$ & $AE \perp CD$ \\
\hline
\end{tabular}

			\caption{Components}
			\label{table:1}
		\end{center}
	\end{table}
\pagebreak

\section{Hardware}
	The IC 7474 is a type of flip-flop integrated circuit that is commonly used in digital electronics applications.It is a dual positive-edge-triggered by the rising edge of a clock signal.Below is the pin diagram of IC 7474:
	\begin{figure}[h]
		\centering
		\begin{figure}[h]
	\begin{center}
	\begin{karnaugh-map}[2][2][1][$R$][$S$]
		\minterms{1,2}
		\autoterms[0]
	\end{karnaugh-map}
	\end{center}	
\caption{k-map}
\label{fig2}
\end{figure}

			\caption{7474}
			\label{fig:2}
		\end{figure}

The connections between arduino and two 7474 IC's and one 7447 IC is:
	\begin{table}[h]
		\begin{center}
	\begin{tabular}{|p{2cm}|p{2cm}|p{2cm}|}
\hline
\multicolumn{3}{|c|}{Truth table}\\
\hline
R& S& A\\
\hline
0& 0& 0\\
\hline
0& 1& 1\\
\hline
1& 0& 1\\
\hline
1& 1& 0\\
\hline
\end{tabular}

			\caption{Arduino-7474}
			\label{table:1}
	\end{center}
	\end{table}

The transition table of the finite state machine is given in the below table \\
\pagebreak
	\begin{table}[h]
	\begin{center}
		\begin{tabular}{|c|c|c|}
\hline
Symbol & Description\\
\hline
c & $\norm{\vec{D} -\vec{C}}$ \\
\hline
r & $\norm{\vec{A} -\vec{D}}$ \\
\hline
d & $\norm{\vec{D} - \vec{E}}$\\
\hline
b & $\norm{\vec{A} - \vec{E}}$\\
\hline
$\theta$ & $\angle{\vec{D}}$ \\
\hline
\end{tabular}

		\caption{Truth table}
		\label{table:3}
	\end{center}
	\end{table}


K-map for A is \\
\begin{figure}[h]
		\centering
		\begin{karnaugh-map}[4][4][1][$Y$][$X$][$I$][$Z$] 
\minterms{7}
\maxterms{0,1,2,3,4,5,6,8,9}
\indeterminants{10,11,12,13,14,15}
\implicant{7}{15}
\end{karnaugh-map}

		\caption{kmap}
		\label{fig:3}
\end{figure}


The resultant expression for $A$ is $ A = YZI $.\\

\pagebreak

K-map for B is \\
\begin{figure}[h]
	\centering
	\begin{karnaugh-map}[4][4][1][$Y$][$X$][$I$][$Z$] 
\minterms{3,4,8}
\maxterms{0,1,2,5,6,7,9}
\indeterminants{10,11,12,13,14,15}
\implicant{4}{12}
\implicantedge{3}{3}{11}{11}
\implicantedge{12}{8}{14}{10}
\end{karnaugh-map}

	\caption{kmap}
	\label{fig:4}
\end{figure}

The resultant expression for $B$ is $ B = YZ'I'+Y'ZI+XI' $.\\

K-map for C is \\
\begin{figure}[h]
	\centering
	\begin{karnaugh-map}[4][4][1][$Y$][$X$][$I$][$Z$] 
\minterms{0,2,4,6,8}
\maxterms{1,3,5,7,9}
\indeterminants{10,11,12,13,14,15}
\implicantedge{0}{8}{2}{10}
\end{karnaugh-map}

	\caption{kmap}
	\label{fig:5}
\end{figure}

The resultant expression for $C$ is $ C = I' $.\\

K-map for D is \\
\begin{figure}[h]
	\centering
	\begin{karnaugh-map}[4][4][1][$Y$][$X$][$I$][$Z$] 
\minterms{8}
\maxterms{0,1,2,3,4,5,6,7,9}
\indeterminants{10,11,12,13,14,15}
\implicantedge{12}{8}{14}{10}
\end{karnaugh-map}

	\caption{kmap}
	\label{fig:6}
\end{figure}

The resultant expression for $D$ is $ D = XI' $.\\

\section{Software}

The assembly code to implement the above finite state machine is: \\
		\lstinputlisting{assignment.asm}
\end{document}

\documentclass{article}
\usepackage[none]{hyphenat}
\usepackage{enumitem}
\usepackage{graphics}
\usepackage{graphicx}
\usepackage{ragged2e}
\usepackage{multirow}
\usepackage{blindtext}
\usepackage{amsmath}
\usepackage{subcaption}
\usepackage{circuitikz}
\usepackage{listings}
\usetikzlibrary{shapes.geometric}
\lstset{
	language=C++,
	basicstyle=\ttfamily\footnotesize,
	breaklines=true,
	frame=lines
	}
\title{Implementation of the below circuit using avr-gcc}
\date{March 2023}
\author{M Sai Sarath Chandra\\chandu.4567890@gmail.com\\FWC22117\\IIT Hyderabad-Future Wireless Communication Assignment}

\begin{document}
\maketitle
	\tableofcontents

\pagebreak

\section{Problem}
	{GATE EC-2019}\\
	Q.25. In the circuit shown,the clock frequency, i.e.,the frequency of the clock signal ,is 12 KHz.The frequency of the signal at Q2 is ............ KHz.
	\begin{figure}[h]
	\centering
		\begin{tikzpicture}
	\draw (0,-10) rectangle (3,-14);
	\draw (5,-10) rectangle (8,-14);
	\draw (-2,-15) -- (4,-15);
	\draw (-1,-15) -- (-1,-13.5);
	\draw (-1,-13.5) -- (0,-13.5);
	\draw (4,-15) -- (4,-13.5);
	\draw (4,-13.5) -- (5,-13.5);
	\draw (-1.5,-10.5) -- (0,-10.5);
	\draw (3,-10.5) -- (5,-10.5);
	\draw (-2,-15) node[above]{$12 KHz$} -- (-1.5,-15);
	\draw (8,-10.5) -- (10,-10.5);
	\draw (0.25,-10.5) node{$D_1$};
	\draw (5.25,-10.5) node{$D_2$};
	\draw (2.75,-10.5) node{$Q_1$};
	\draw (2.75,-13.5) node{$Q_1'$};
	\draw (7.75,-10.5) node{$Q_2$};
	\draw (7.75,-13.5) node{$Q_2'$};
	\draw (0.30,-13.5) node{$Clk$};
	\draw (5.30,-13.5) node{$Clk$};
	\node[and port] (a) at (-1.5,-10.5){};
	\draw (3,-13.5) -- (3.5,-13.5);
	\draw (3.5,-13.5) -- (3.5,-9.75);
	\draw (3.5,-9.75) -| (a.in 1);
	\draw (8,-13.5) -- (8.5,-13.5);
	\draw (8.5,-13.5) -- (8.5,-9.5);
	\draw (8.5,-9.5) -- (-3.5,-9.5);
	\draw (-3.5,-9.5) -- (-3.5,-10.78);
	\draw (-3.5,-10.78) -- (a.in 2);
\end{tikzpicture}

		\caption{circuit}
		\label{fig:1}
	\end{figure}

\section{Introduction}
		
		The aim is to implement the above sequential circuit using D flip-flops (IC 7474) and to find out the frequency of the signal at Q2(it is given that the frequency of the clock signal is 12KHz).IC 7474 is a dual positive edge triggered D type flip flop,which means it has two separate flip-flop that are triggered by the rising edge of a clock signal.

		In the above circuit $Q_1$,$Q_2$ are inputs and $D_1$,$D_2$ are outputs.So,from the circuit the expressions of $D_1$ and $D_2$ are:

		$D_1 = Q_1'Q_2'$.\\
			$D_2 = Q_1$.\\

Below is the transition table of the above circuit which is as follows:
\pagebreak

	\begin{table}[h]
		\begin{center}
			\begin{tabular}{|c|c|c|}
\hline
Point & Co-ordinates & Description\\
\hline
D & (0,0) & Origin(Assumption) \\
\hline
C & (16,0) & (because $\vec{CD} = 16cm$)\\
\hline
E & (10,0) & (because DE = AE * $\cot{D}$)\\
 & & DE = 8 * $\cot(38.68)$ = 10cm \\
\hline
A & (10,8) & As A lies above E ($A_x = E_x$)\\
	& & because $\vec{AE} = 8cm$ ($A_y$ = 8)\\
\hline
B & (26,8) & (because = $\vec{AB} = 16cm$)\\
\hline
F & (9.75,7.806) & Using distance formula\\
	& & for $\vec{CF}$ and $\vec{DF}$ and solving them.\\
\hline
\end{tabular}

			\caption{Transition table}
			\label{table:2}
		\end{center}
	\end{table}

\section{Components}
	
	\begin{table}[h]
		\begin{center}
			\begin{tabular}{|c|c|c|}
\hline
Symbol & Value & Description\\
\hline
$x$ & 16cm & $\vec{AB}$ \\
\hline
$a$ & 10cm & $\vec{CF}$ \\
\hline
$b$ & 8cm & $\vec{AE}$ \\
\hline
$\angle{CFD}$ & $90\degree$ & $CF \perp AD$ \\
\hline
$\angle{AED}$ & $90\degree$ & $AE \perp CD$ \\
\hline
\end{tabular}

			\caption{Components}
			\label{table:1}
		\end{center}
	\end{table}


\section{Hardware}

	IC 7474 is a D flip-flop integrated circuit that is commonly used in digital electronics applications.It is a dual positive edge-triggered by the rising edge of a clock signal.Below is the pin diagram of IC 7474:
	\begin{figure}[h]
		\centering
			\begin{center}
	\begin{karnaugh-map}[2][2][1][$R$][$S$]
		\minterms{1,2}
		\autoterms[0]
	\end{karnaugh-map}
	\end{center}	

		\caption{7474}
		\label{fig:2}
	\end{figure}


The connections between the arduino and IC 7474 is as follows:
	\begin{table}[h]
		\begin{center}
			\begin{tabular}{|c|c|c|}
\hline
Symbol & Description\\
\hline
c & $\vec{DC}$ \\
\hline
r & $\vec{AD}$ \\
\hline
d & $\vec{DE}$\\
\hline
b & $\vec{AE}$\\
\hline
$\theta$ & $\angle{\vec{D}}$ \\
\hline
\end{tabular}

			\caption{connections}
			\label{table:3}
		\end{center}
	\end{table}


\section{Software}

The code to implement the above circuit is : \\

		\lstinputlisting{main.c}

\end{document}

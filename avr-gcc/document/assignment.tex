\documentclass{article}
\usepackage[none]{hyphenat}
\usepackage{enumitem}
\usepackage{graphics}
\usepackage{graphicx}
\usepackage{ragged2e}
\usepackage{multirow}
\usepackage{blindtext}
\usepackage{amsmath}
\usepackage{subcaption}
\usepackage{circuitikz}
\usepackage{listings}
\usetikzlibrary{shapes.geometric}
\lstset{
	language=C++,
	basicstyle=\ttfamily\footnotesize,
	breaklines=true,
	frame=lines
	}
\title{Implementation of the below circuit using avr-gcc}
\date{March 2023}
\author{M Sai Sarath Chandra\\chandu.4567890@gmail.com\\FWC22117\\IIT Hyderabad-Future Wireless Communication Assignment}

\begin{document}
\maketitle
	\tableofcontents

\pagebreak

\section{Problem}
	{GATE EC-2019}\\
	Q.25. In the circuit shown,the clock frequency, i.e.,the frequency of the clock signal ,is 12 KHz.The frequency of the signal at Q2 is ............ KHz.
	\begin{figure}[h]
	\centering
		\begin{figure}[h]
	\begin{center}
\begin{tikzpicture}
\ctikzset
{
logic ports=ieee,
logic ports/scale=0.8
}
\draw(2,1) -- (4,1);
\draw(2,0) -- (4,0);
\draw(2,-1) -- (4,-1);
\draw(2,-2) -- (4,-2);
\draw(4,-3) -- (4,2);
\draw(4,2) -- (7,2);
\draw(7,2) -- (7,-3);
\draw(7,-3) -- (4,-3);
\draw(5,-3) -- (5,-5);
\draw(6,-3) -- (6,-5);
\draw(7,-0.5) -- (9,-0.5);
\draw(1.5,1) node{$W$};
\draw(1.5,0) node{$X$};
\draw(1.5,-1) node{$Y$};
\draw(1.5,-2) node{$Z$};
\draw(5,-5.5) node{$R$};
\draw(6,-5.5) node{$S$};
\draw(9.5,-0.5) node{$A$};
\end{tikzpicture}
	\end{center}
\caption{mux}
\label{fig:1}
\end{figure}

		\caption{circuit}
		\label{fig:1}
	\end{figure}

\section{Introduction}
		
		The aim is to implement the above sequential circuit using D flip-flops (IC 7474) and to find out the frequency of the signal at Q2(it is given that the frequency of the clock signal is 12KHz).IC 7474 is a dual positive edge triggered D type flip flop,which means it has two separate flip-flop that are triggered by the rising edge of a clock signal.

		In the above circuit $Q_1$,$Q_2$ are inputs and $D_1$,$D_2$ are outputs.So,from the circuit the expressions of $D_1$ and $D_2$ are:

		$D_1 = Q_1'Q_2'$.\\
			$D_2 = Q_1$.\\

Below is the transition table of the above circuit which is as follows:
\pagebreak

	\begin{table}[h]
		\begin{center}
			\begin{tabular}{|p{2cm}|p{2cm}|p{2cm}|}
\hline
\multicolumn{3}{|c|}{Truth table}\\
\hline
R& S& A\\
\hline
0& 0& 0\\
\hline
0& 1& 1\\
\hline
1& 0& 1\\
\hline
1& 1& 0\\
\hline
\end{tabular}

			\caption{Transition table}
			\label{table:2}
		\end{center}
	\end{table}

\section{Components}
	
	\begin{table}[h]
		\begin{center}
			\begin{tabular}{|c|c|c|}
\hline
Symbol & Value & Description\\
\hline
$x$ & 16cm & $\vec{AB}$ \\
\hline
$a$ & 10cm & $\vec{CF}$ \\
\hline
$b$ & 8cm & $\vec{AE}$ \\
\hline
$\angle{CFD}$ & $90\degree$ & $CF \perp AD$ \\
\hline
$\angle{AED}$ & $90\degree$ & $AE \perp CD$ \\
\hline
\end{tabular}

			\caption{Components}
			\label{table:1}
		\end{center}
	\end{table}


\section{Hardware}

	IC 7474 is a D flip-flop integrated circuit that is commonly used in digital electronics applications.It is a dual positive edge-triggered by the rising edge of a clock signal.Below is the pin diagram of IC 7474:
	\begin{figure}[h]
		\centering
		\begin{figure}[h]
	\begin{center}
	\begin{karnaugh-map}[2][2][1][$R$][$S$]
		\minterms{1,2}
		\autoterms[0]
	\end{karnaugh-map}
	\end{center}	
\caption{k-map}
\label{fig2}
\end{figure}

		\caption{7474}
		\label{fig:2}
	\end{figure}


The connections between the arduino and IC 7474 is as follows:
	\begin{table}[h]
		\begin{center}
			\begin{tabular}{|c|c|c|}
\hline
Symbol & Description\\
\hline
c & $\norm{\vec{D} -\vec{C}}$ \\
\hline
r & $\norm{\vec{A} -\vec{D}}$ \\
\hline
d & $\norm{\vec{D} - \vec{E}}$\\
\hline
b & $\norm{\vec{A} - \vec{E}}$\\
\hline
$\theta$ & $\angle{\vec{D}}$ \\
\hline
\end{tabular}

			\caption{connections}
			\label{table:3}
		\end{center}
	\end{table}


\section{Software}

The code to implement the above circuit is : \\

		\lstinputlisting{main.c}

\end{document}

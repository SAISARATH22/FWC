\documentclass[12pt]{article}
\usepackage{graphicx}
\usepackage{gensymb}
\usepackage[none]{hyphenat}
\usepackage{graphicx}
\usepackage{listings}
\usepackage[english]{babel}
\usepackage{graphicx}
\usepackage{caption}
\usepackage{hyperref}
\usepackage{booktabs}
\usepackage{array}
\usepackage{amsmath}
\usepackage{listings}
\usepackage{multirow}
\usepackage{blindtext}
\usepackage{capt-of}
\usepackage{circuitikz}
\usetikzlibrary{shapes.geometric}
\title{Implementation of 4x1 mux in Arduino using ICs}
\date{February 2023}
\lstset{
	frame=single'
	breaklines=true
}
\newcommand{\mydet}[1]{\ensuremath{\begin{vmatrix}#1\end{vmatrix}}}
\providecommand{\brak}[1]{\ensuremath{\left(#1\right)}}
\providecommand{\norm}[1]{\left\lvert#1\right\rVert}
\newcommand{\solution}{\noindent \textbf{Solution: }}
\newcommand{\myvec}[1]{\ensuremath{\begin{pmatrix}#1\end{pmatrix}}}
\let\vec\mathbf

\begin{document}
\maketitle 
	$\textbf{Contents:}$
\begin{enumerate}
	\item Problem
	\item Introduction
	\item Components
	\item Hardware
	\item Software
\end{enumerate}
\begin{enumerate}
	\pagebreak \item $\textbf{Problem}$
		\newline(GATE EC-2022)\newline

Q.19. Consider the 2-bit multiplexer(MUX) shown in the figure.For output to be the XOR of R and S,the values for $ W,X,Y$ and $Z$ are ?\newline
\begin{tikzpicture}
\ctikzset
{
logic ports=ieee,
logic ports/scale=0.8
}
\draw(2,1) -- (4,1);
\draw(2,0) -- (4,0);
\draw(2,-1) -- (4,-1);
\draw(2,-2) -- (4,-2);
\draw(4,-3) -- (4,2);
\draw(4,2) -- (7,2);
\draw(7,2) -- (7,-3);
\draw(7,-3) -- (4,-3);
\draw(5,-3) -- (5,-5);
\draw(6,-3) -- (6,-5);
\draw(7,-0.5) -- (9,-0.5);
\draw(1.5,1) node{$W$};
\draw(1.5,0) node{$X$};
\draw(1.5,-1) node{$Y$};
\draw(1.5,-2) node{$Z$};
\draw(5,-5.5) node{$R$};
\draw(6,-5.5) node{$S$};
\draw(9.5,-0.5) node{$A$};
\end{tikzpicture}
\begin{enumerate}
\item $W = 0, X = 0, Y = 1, Z = 1$
\item $W = 1, X = 0, Y = 1, Z = 0$
\item $W = 0, X = 1, Y = 1, Z = 0$
\item $W = 1, X = 1, Y = 0, Z = 0$
\end{enumerate}
\item $\textbf{Introduction}$ \newline
	The above diagrm is a 4:1 multiplexer where $W, X, Y, Z$ are the inputs of the multiplexer and $A$ is the output of the multiplexer.$R , S$ are the select lines of the multiplexer,which means:\newline
\begin{enumerate}
\item For $R = 0,S = 0$,the first input line $W$ is selected.
\item For $R = 0,S = 1$,the second input line $X$ is selected.
\item For $R = 1,S = 0$,the third input line $Y$ is selected.
\item For $R = 1,S = 1$,the fourth input line $Z$ is selected.
\end{enumerate}
Therefore,the resultant output expression of the multiplexer is $R'S'W + R'SX + RS'Y + RSZ$.
\item $\textbf{Components}$
\captionof{table}{Table1}\label{table:1}\begin{tabular}{|p{5cm}|p{3cm}|p{2cm}|}
	\hline
	\multicolumn{3}{|c|}{COMPONENTS}\\
	\hline
	Component& Value& Quantity\\
	\hline
	Resistor& 220 ohm& 1\\
	\hline
	Arduino& UNO& 1\\
	\hline
	Seven Segment Display& & 1\\
	\hline
	Jumper Wires& M-M& 20\\
	\hline
	Breadboard& & 1\\
	\hline
\end{tabular}
\item $\textbf{Hardware}$
	\begin{enumerate}
\item Connect the COM of the seven-segment display to 5V and dot of the seven-segment to the ground.
\item Now connect any one of the pin of the seven-segment to pin no.2(digital).
\item Pin no.s 5,6,7,8 of the arduino should be initially connected to ground.
\item Now move pin no.s 5,6,7,8 accordingly and for the right combination the second pin of the arduino becomes high and the seven segement display glows.
	\end{enumerate}
\item $\textbf{Software}$
	\newline
The code below can help in solving the above problem.
\lstinputlisting{q19.cpp}
\end{enumerate}
\end{document}

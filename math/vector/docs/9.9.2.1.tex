\documentclass{article}
\usepackage{amsmath}
\usepackage{xcolor}
\usepackage{gensymb}
\usepackage{ragged2e}
\usepackage{graphicx}
\usepackage{gensymb}
\usepackage{mathtools}
\newcommand{\mydet}[1]{\ensuremath{\begin{vmatrix}#1\end{vmatrix}}}
\providecommand{\brak}[1]{\ensuremath{\left(#1\right)}}
\providecommand{\norm}[1]{\left\lVert#1\right\rVert}
\newcommand{\solution}{\noindent \textbf{Solution: }}
\newcommand{\myvec}[1]{\ensuremath{\begin{pmatrix}#1\end{pmatrix}}}
\let\vec\mathbf
\begin{document}
\begin{center}
        \textbf\large{CHAPTER-9 \\ AREAS OF PARALLELOGRAMS AND TRIANGLES}
\end{center}
\section{Exercise 9.2}
Q1. In the figure given below, $ABCD$ is a parallelogram, $AE \perp DC$ and $CF \perp AD$.If $AB = 16cm$, $AE = 8cm$ and $CF = 10cm$, find $AD$.\\
\textbf{Construction}\\
\begin{figure}[h]
 \begin{center}
  \includegraphics[width=\columnwidth]{fig1.png}
 \end{center}
 \caption{Parallelogram ABCD}
 \label{fig:Fig}
\end{figure}\\
\pagebreak
The following table consists of given input parameters of the above parallelogram ABCD :\\
\begin{table}[h]
\centering
	\begin{tabular}{|c|c|c|}
\hline
Symbol & Value & Description\\
\hline
$x$ & 16cm & $\vec{AB}$ \\
\hline
$a$ & 10cm & $\vec{CF}$ \\
\hline
$b$ & 8cm & $\vec{AE}$ \\
\hline
$\angle{CFD}$ & $90\degree$ & $CF \perp AD$ \\
\hline
$\angle{AED}$ & $90\degree$ & $AE \perp CD$ \\
\hline
\end{tabular}

	\caption{Parameters}
	\label{tab:table1}
\end{table}\\
Table below has the given input co-ordinates of the parallelogram :\\
\begin{table}[h]
	\centering
	\begin{tabular}{|c|c|c|}
\hline
Point & Co-ordinates & Description\\
\hline
D & (0,0) & Origin(Assumption) \\
\hline
C & (16,0) & (because $\vec{CD} = 16cm$)\\
\hline
E & (10,0) & (because DE = AE * $\cot{D}$)\\
 & & DE = 8 * $\cot(38.68)$ = 10cm \\
\hline
A & (10,8) & As A lies above E ($A_x = E_x$)\\
	& & because $\vec{AE} = 8cm$ ($A_y$ = 8)\\
\hline
B & (26,8) & (because = $\vec{AB} = 16cm$)\\
\hline
F & (9.75,7.806) & Using distance formula\\
	& & for $\vec{CF}$ and $\vec{DF}$ and solving them.\\
\hline
\end{tabular}

	\caption{Co-ordinates}
	\label{tab:table2}
\end{table}\\
Following table are the unknown lengths and angles and their symbols: \\
\begin{table}[h]
	\centering
	\begin{tabular}{|c|c|c|}
\hline
Symbol & Description\\
\hline
c & $\vec{DC}$ \\
\hline
r & $\vec{AD}$ \\
\hline
d & $\vec{DE}$\\
\hline
b & $\vec{AE}$\\
\hline
$\theta$ & $\angle{\vec{D}}$ \\
\hline
\end{tabular}

	\caption{Symbols and Corresponding Vectors}
	\label{tab:table3}
\end{table}\\
The point co-ordinates are derived in the following way : \\
\begin{enumerate}
	\item \textbf{To derive the co-ordinates of C :}\\
		As mentioned in the \ref{tab:table3}, $\norm{\vec{D} - \vec{C}} = c$.In the above parallelogram it is given that $\norm{\vec{B} - \vec{A}} = 16cm$.According to the properties of a parallelogram the parallel sides are equal in length.So, it can be said that : \\
		\begin{align}
			\norm{\vec{B} - \vec{A}} = \norm{\vec{D} - \vec{C}} = c\\
		\end{align}
As point $\vec{C}$ lies on x axis, it can be expressed in the following way :\\
		\begin{align}
			\vec{C} = c\vec{e_1}\\
			\vec{C} = c\myvec{1\\0} = \myvec{c\\0}\\
		\end{align}
	\item \textbf{To derive the co-ordinates of A :}\\
		A can be expressed in the form of $r\myvec{\cos{\theta}\\\sin{\theta}}$.In order to obtain r and $\theta$, the following can be done : \\
		\begin{enumerate}
			\item \textbf{To find out $\theta$ :}\\
		To find out $\theta$,let us assume that $\norm{\vec{C} - \vec{F}} = a$		
		\begin{align}
			from \triangle{CFD},\\
			\sin{\theta} = \frac{a}{c}\\
			\implies \theta = \sin^{-1}{\frac{a}{c}} \\
		\end{align}
	\item \textbf{To find out r :}\\
		As mentioned in \ref{tab:table3}, $\norm{\vec{D} - \vec{A}} = r$ and $\norm{\vec{E} - \vec{A}} = b$.In order to find out $r$,
		\begin{align}
			from \triangle{ADE},\\
			\sin{\theta} = \frac{b}{r}\\
			r = \frac{b}{\sin{\theta}}\\
		\end{align}
		So, the co-ordinates of $\vec{A}$ can be written as :\\
		\begin{align}
			\vec{A} = \frac{b}{\sin{\theta}}\myvec{\cos{\theta}\\\sin{\theta}}\\
			\vec{A} = \myvec{b\cot{\theta}\\b}\\
		\end{align}
		\end{enumerate}
	\item \textbf{To derive the co-ordinates of B :}\\
		From parallelogram law of vectors, $\vec{B}$ can be expressed as the sum of $\vec{A}$ and $\vec{C}$.So, it can be written as,\\
		\begin{align}
			\vec{B} = \vec{A} + \vec{C}\\
		\end{align}
	\item \textbf{To derive the co-ordinates of E :}\\
		As mentioned in the table3, $\norm{\vec{D} - \vec{E}} = d$.As, $\vec{E}$ lies on x-axis it can be written in the form of $de_1$.So, the co-ordinates can be found out in the following way : \\
		\begin{align}
			from \triangle{DAE},\\
			\cos{\theta} = \frac{d}{r}\\
			d = r\cos{\theta}\\
		\end{align}
		 $\vec{E}$ = $d\vec{e_1}$ = r$\cos{\theta}\myvec{1\\0}$ = $\myvec{r\cos{\theta}\\0}$.\\
	 \item \textbf{To derive the co-ordinates of F :}\\
		 As point $\vec{F}$ divides $\vec{AD}$ in the ratio k : 1.The co-ordinates of $\vec{F}$ can be found out in the following way : \\
		 \begin{align}
			 \vec{F} = \frac{k\vec{A} + \vec{D}}{k + 1}\\
		 \end{align}
\end{enumerate}
The following table displays the unknown lengths and angles which were derived from the given quantities :\\
\begin{table}[h]
	\centering
	\begin{tabular}{|c|c|c|}
\hline
Symbol & value & Description\\
\hline
c & x & $\norm{\vec{D} - \vec{C}}$\\
\hline
r & $\frac{b}{\sin{\theta}}$ & $\norm{\vec{D} - \vec{A}}$ \\
\hline
d & r$\cos{\theta}$ & $\norm{\vec{D} - \vec{E}}$ \\
\hline
$\theta$ & $\sin^{-1}\frac{a}{c}$ & $\angle{D}$ \\
\hline
\end{tabular}


	\caption{Co-ordinates in terms of given and derived lengths and angles}
	\label{tab:table4}
\end{table}\\
\pagebreak
The following table displays the point co-ordinates in terms of known and derived quantities:\\
\begin{table}[h]
	\centering
	\begin{tabular}{|c|c|c|}
\hline
Point & Co-ordinates\\
\hline
$\vec{A}$ & $\myvec{b\cot{\theta}\\b}$\\
\hline
$\vec{B}$ & $\vec{A} + \vec{C}$\\
\hline
$\vec{C}$ & $\myvec{c\\0}$\\
\hline
$\vec{E}$ & $\myvec{r\cos{\theta}\\0}$\\
\hline
$\vec{F}$ & $\frac{k\vec{A} + \vec{D}}{k + 1}$\\
\hline
\end{tabular}

	\caption{Co-ordinates in terms of known and derived quantities}
	\label{tab:table5}
\end{table}\\
\textbf{Finding the co-ordinates of the above parallelogram:}\\
\begin{enumerate}
	\item \textbf{Co-ordinates of $\vec{C}$:}\\
		As c = $\norm{\vec{D} - \vec{C}} = 16$, the co-ordinates of $\vec{C}$ are:
		\begin{align}
			\vec{C} = \myvec{c\\0} = \myvec{16\\0}\\
		\end{align}
So, the co-ordinates of $\vec{C}$ are $\myvec{16\\0}$.
\item \textbf{Co-ordinates of $\vec{A}$:}\\
	\begin{enumerate}
		\item \textbf{Finding $\theta$:}\\
			\begin{align}
				\theta = \sin^{-1}\frac{a}{c}\\
				\theta = \sin^{-1}\frac{10}{16}\\
				\theta = 38.68\degree\\
			\end{align}
	\end{enumerate}
From the above derivations, we got that:\\
			\begin{align}
				\vec{A} = \myvec{b\cot{\theta}\\b}\\
				\implies \myvec{8\cot{38.68}\\8}\\
				\vec{A} = \myvec{10\\8}\\
			\end{align}
so, the co-ordinates of $\vec{A}$ are $\myvec{10\\8}$.
\item \textbf{Co-ordinates of $\vec{B}$:}\\
	From above derivation, we got that $\vec{B} = \vec{A} + \vec{C}$.Then, $\vec{C}$ is:\\
		\begin{align}
			\vec{B} = \myvec{10\\8} + \myvec{16\\0}\\
			\vec{B} = \myvec{26\\8}\\
		\end{align}
So, the co-ordinates of $\vec{B}$ are $\myvec{26\\8}$.
\item \textbf{Co-ordinates of $\vec{E}$:}\\
	\begin{enumerate}
		\item \textbf{Finding r:}\\
From above derivation, r can be found out in the following way:\\
			\begin{align}
				r = \frac{b}{\sin{\theta}}\\
				r = \frac{8}{\sin{38.68}}\\
				r = 12.8cm\\
			\end{align}
	\end{enumerate}
		The co-ordinates of $\vec{E}$ are:
		\begin{align}
			\vec{E} = \myvec{r\cos{\theta}\\0}\\
			\implies \myvec{(12.8)\cos{38.68}\\0}\\
			\vec{E} = \myvec{10\\0}\\
		\end{align}
So, the co-ordinates for $\vec{E}$ are $\myvec{10\\0}$.
\item \textbf{Co-ordinates for $\vec{F}$:}\\
	$\vec{F}$ divides $\vec{AD}$ in the ratio 39 : 1.So, the co-ordinates of $\vec{F}$ are:
		\begin{align}
			\vec{F} = \frac{k\vec{A} + \vec{D}}{k + 1}\\
			\implies \frac{(39)\myvec{10\\8} + \myvec{0\\0}}{39 + 1}\\
			\implies \frac{\myvec{390\\312}}{40}\\
			\vec{F} = \myvec{9.75\\7.8}
		\end{align}
So, the co-ordinates of $\vec{F}$ are $\myvec{9.75\\7.8}$.
\end{enumerate}
The following table displays the final co-ordinates of the vertices of the parallelogram :\\
\begin{table}[h]
	\centering
	\input{tables/table6.tex}
	\caption{Final co-ordinates of the parallelogram}
	\label{tab:table6}
\end{table}\\
The length of $\norm{\vec{D} - \vec{A}}$ was found out in the above process and it is r = 12.8cm.
\end{document}
